\documentclass{acmtrans2m}

\usepackage[utf8x]{inputenc}
\usepackage[english]{babel}
\usepackage[T1]{fontenc} 
\usepackage{amsmath}
\usepackage{algorithm2e}
\usepackage{color}
\usepackage{listings} 
\newcommand{\hervorheben}{\color{red}}%
\newenvironment{algorithmic}{%
\renewenvironment{algocf}[1][h]{}{}% pass over the floating stuff
\algorithm
}{%
\endalgorithm
}
\usepackage[pdftex]{graphicx}

\newtheorem{theorem}{Theorem}[section]
\newtheorem{conjecture}[theorem]{Conjecture}
\newtheorem{corollary}[theorem]{Corollary}
\newtheorem{proposition}[theorem]{Proposition}
\newtheorem{lemma}[theorem]{Lemma}
\newdef{definition}[theorem]{Definition}
\newdef{remark}[theorem]{Remark}
           
\markboth{Dennis Hoppe}{DARXplorer 2.0 -- Developer Documentation}

\title{DARXplorer 2.0 -- Developer Documentation}
            
\author{DENNIS HOPPE\\
      Bauhaus-University Weimar 
}

\lstset{language=Ada,numbers=none,frame=none,basicstyle=\footnotesize,breaklines,tabsize=2,captionpos=b,prebreak={\hbox{ $\rightarrow$}},postbreak={\hbox{$\hookrightarrow$ }},showstringspaces=false}
            
\begin{abstract}
DARXplorer is a toolbox for cryptanalysts and cipher designers. It
supports the differential cryptanalysis of primitives which use a mixture of addition, 
rotation and xor. DARXplorer attempts to find good xor differences. The two main 
use cases for DARXplorer are (1) aiding the designers of new primitives to defend 
against attacks and (2) analyzing existing primitives to find attacks.
\end{abstract}

\category{...}{...}{...}
%\terms{...}             
%\keywords{...}
      
\begin{document}
    
\maketitle

%
% EINLEITUNG
%
\section{Introduction}
\label{section:einleitung}

\textit{DARXplorer} was one of the MSL laboratory projects in the summer term 2008.
Goal of the project was to create a framework supporting the development of the block 
cipher \textit{Threefish} \cite{Anonymous:2008p2963} used in the hash function \textit{Skein} \cite{Whiting:2009p11767}. Threefish based on a
mixture of addition modulo $2^{64}$, rotation and xor. DARXplorer attempts to find good xor differences.

Besides supporting the development of new primitives, DARXplorer's second goal is to find attacks against existing primitives like \textit{SHA-256}. For that reason, DARXplorer
was enhanced in a consecutive research project in the summer term 2009 to allow the analyzation of arbitrary primitives so long as they use a mixture of the previous stated mixture of operations. Due to the fact, that the framework was completely rewritten, this 
developer documentation will give you detailed information about the 
implementation for further development.

For more information regarding DARXplorer in general, see \citeN{Lucks:2008p2672}.

\bigbreak

This developer documentation is about installation and usage of DARXplorer in chapters \ref{chapter:pre} and \ref{chapter:installation}. Chapter \ref{chapter:project} presents the object
structure in detail. It is shown, which steps are needed to integrate an input source file into
DARXplorer. Chapter \ref{chapter:syntax} attends the supported syntax for writing
declarations of cryptographic primitives. Subsequent to this chapter discusses chapter
\ref{chapter:techniques} all implemented techniques in detail and shows how to
add new techniques. Chapter \ref{chapter:log} deals with user-defined formatting
of log files, before in chapter \ref{chapter:future} challenges and future work are discussed.

\section{Prerequisites}\label{chapter:pre}
All you need is \textit{GNAT}, the free-software compiler for Ada in version 4.3 
or higher. If you are running an UNIX based operating system like Ubuntu, you can
get the current GNAT compiler simply by typing in your preferred terminal:\\ \\
\noindent\hspace*{12mm}%
\texttt{\$ sudo apt-get install gnats}\\
\noindent\hspace*{12mm}%
\texttt{\$ gnatmake --version}\\ \\
You should see the following output:\\ \\
\noindent\hspace*{12mm}%
\texttt{GNATMAKE 4.3.4}\\
\noindent\hspace*{12mm}%
\texttt{Copyright (C) 1995-2007, Free Software Foundation, Inc.}\\
\noindent\hspace*{12mm}%
\texttt{This is free software; see the source for copying conditions.}\\
\noindent\hspace*{12mm}%
\texttt{There is NO warranty; not even for MERCHANTABILITY or}\\ 
\noindent\hspace*{12mm}%
\texttt{FITNESS FOR A PARTICULAR PURPOSE.}\\

DARXplorer does not use any GNAT-specific features. It is likely, that
DARXplorer will compile with GNAT versions prior to 4.3. We have tested
the current distribution successfully under \textit{Ubuntu} 9.04 with GNAT 4.3 installed.
Further on, DARXplorer is running under \textit{MAC OS X} (Version 10.5.x)
with GNAT GPL 2009 installed. If you use MAC OS X 10.6 (Snow Leopard),
you have to upgrade to an experimental branch of GNAT 4.4, because GNAT GPL 2009
is currently not running on Snow Leopard.

For the development of DARXplorer are no special tools required.

\section{Installation}\label{chapter:installation}

After the successful installation of GNAT you are ready to download DARXplorer.
Checkout the sources from the repository to your local folder by typing\\ \\
\noindent\hspace*{12mm}%
\texttt{\$ svn co svn://medsec.medien.uni-weimar.de/darx}\\ \\
Please have your username and password on tap.

In the checkout directory, you will find a \texttt{Makefile} and a shell script named \texttt{build.sh}. For development purposes, the Makefile is sufficient. The shell script offers a convenient way for end users to use DARXplorer.

\subsection{Makefile}
The \texttt{Makefile} offers you the following targets:
\begin{enumerate}
	\item \texttt{compiler} Compiles the DARXplorer framework and creates the executable \texttt{dxpl-compiler-cli}, that can be used to convert primitive declarations in Ada. While transforming the description, several files will be added to the framework.
	\item \texttt{run-compiler} Usually, if you went through step (1), you have to call the executable with your desired source file. The target \texttt{run-compiler} incorporates both steps by taking the script as an argument, e.g. : \texttt{make run-compiler ARG=examples/threefish256.adb}. This target returns \texttt{TRUE}, if no errors occurred.
	\item \texttt{function} Additional files created in step (1) or (2) have to be compiled a second time. This target generates the \texttt{darxplorer} executable, finally.
	\item \texttt{clean} Cleans up the object directory.
\end{enumerate}

All in all, if you want to create a DARXplorer instance incorporating Threefish 256, you have to execute the following commands in the root of the distribution:\\ \\
\noindent\hspace*{12mm}%
\texttt{\$ make run-compiler ARG=examples/Threefish256.adb}\\
\noindent\hspace*{12mm}%
\texttt{\$ make function}\\

\subsection{Shell script}
The shell script applies the same tasks as the \texttt{Makefile}, but provides a simplified interface to allow end users to use DARXplorer in a more convenient way. Only one call to \texttt{build.sh} is sufficient to build the final executable:\\ \\ 
\noindent\hspace*{12mm}%
\texttt{\$ ./build.sh examples/Threefish256.adb}\\ \\
By calling the shell script without arguments or by means of \texttt{-\,-usage} as a parameter, it will display some useful help on the command line.

\section{Project Structure}\label{chapter:project}
DARXplorer currently shows the following project structure, if directly checked out from the repository:\\ \\
\noindent\hspace*{12mm}%
\texttt{\ \ \ \ {\bf trunk}\\
\noindent\hspace*{12mm}%
\ \ \ \ ..{\bf bc-20090226}\\
\noindent\hspace*{12mm}%
\ \ \ \ ..build.sh\\
\noindent\hspace*{12mm}%
\ \ \ \ ..Makefile\\
\noindent\hspace*{12mm}%
\ \ \ \ ..{\bf examples}\\
\noindent\hspace*{12mm}%
\ \ \ \ ..{\bf src}\\
\noindent\hspace*{12mm}%
\ \ \ \ ....{\bf compiler}\\
\noindent\hspace*{12mm}%
\ \ \ \ ......{\bf lexer}\\
\noindent\hspace*{12mm}%
\ \ \ \ ......{\bf parser}\\
\noindent\hspace*{12mm}%
\ \ \ \ ........{\bf support}\\
\noindent\hspace*{12mm}%
\ \ \ \ ....{\bf function}\\
\noindent\hspace*{12mm}%
\ \ \ \ ..{\bf tests}\\
}

Consequent sections will enlighten the contents of each folder.

\subsection{bc-20090226}
We use some data structures provided by the \textit{Ada95 Booch Components}, especially graphs and queues. These data structures are not provided by Ada itself. Unfortunately, the implementations of Ada95 Booch needed some revision by ourselves to offer the behavior, we required. You can find the adapted priority queue in the packages\\ \\
\noindent\hspace*{12mm}%
\texttt{bc-containers-queues-ordered-bounded.ads} and \\
\noindent\hspace*{12mm}%
\texttt{bc-containers-queues-ordered-bounded.adb}\\ \\
and the aligned directed graph in the packages\\ \\
\noindent\hspace*{12mm}%
\texttt{bc-graphs-directed-sorted.ads} and \\
\noindent\hspace*{12mm}%
\texttt{bc-graphs-directed-sorted.adb}\\ \\
in the \texttt{src} folder. DARXplorer is interfacing the Ada95 Booch components by means of the \texttt{dxpl-booch-*} packages, which can be found in the source folder, as well.

The components are updated frequently by two very responsive authors. So it might be a good idea to check their repository once in a while. The interfaces in DARXplorer and the customized versions of the priority queue and the directed graph should not be touched by updates. Just replace the folder \texttt{bc-20090226} with a new version and match the path to the components in the \texttt{Makefile} plus the shell script.

\subsection{examples}
The \texttt{examples} folder contains declarations of cryptographic primitives in Ada, that are based on a mixture of addition, rotation and xor. It is a good place to be started, if you want to learn, how to write descriptions by yourself. At present, we support the following primitives:
\begin{itemize}
	\item \ Blake-32
	\item \ Blake-64
	\item \ Blue Midnight Wish 256
	\item \ Blue Midnight Wish 512
	\item \ Edon-R 256
	\item \ Edon-R 512
	\item \ Salsa-20
	\item \ SHA-256
	\item \ SHA-512
	\item \ Tiny Encryption Algorithm (TEA)
	\item \ Threefish-256
	\item \ Threefish-512
	\item \ Threefish-1024
\end{itemize}

Both \textit{Blue Midnight Wish} and \textit{Threefish} are also covered with revised versions including changes made by submitting tweaked primitives to the second round of the NIST competition.

\subsection{src}
Contains the source distribution of DARXplorer. It includes the following packages:\\ \\
\medskip
\begin{tabular}{l l l}
        Packages & Description\\
\hline
        \verb|dxpl-io-*| & Implementing a project wide debugger, logger\\
        & and formatter.\\
        \verb|dxpl-round_function-*| & Interfaces and basic methods of the round function.\\
        \verb|dxpl-support-*| & Implementation of algorithms provided by\\
        &  Lipmaa and Moriai to examine the differential\\
        &  properties of addition.\\
        \verb|dxpl-technique-*| & Providing implementations of techniques: \\
        & Lazy Laura, Greedy Grete, Pedantic Petra and\\
        & Fuzzy Fiona.\\
        \verb|dxpl-toolbox-*| & Analyzation framework for each technique\\
\end{tabular}

\subsection{src/compiler}
The compiler is used to integrate a description of a cryptographic primitive into the
framework. Therefore, several files are automatically generated and copied to the subdirectory \texttt{src/function}.

The subsequent table lists important packages and files.\\ \\
\medskip
\begin{tabular}{l l l}
        Packages/Files & Description\\
\hline
        \verb|dxpl_types.dummy| & Declaration of project wide data types. This file\\
        & is used to generate automatically consistent Ada sources\\
        & declaring data types based on 32 and 64 bit.\\
        \verb|dxpl-compiler-cli| & Implementation of the command line interface\\
        & of DARXplorer.\\
        \verb|dxpl-compiler-*| & We provided a separate debug package for\\
        & the compiler. Also, implementations of a symbol table\\
        & and a syntax-tree can be found here.
\end{tabular}

The syntax tree is crucial in DARXplorer. It is used to parse arbitrary mathematical
expressions in a primitive description. See chapter \ref{chapter:syntax}.

\subsection{src/compiler/lexer}
A very efficient implementation of a lexer is implemented. The vocabulary,
the lexer and compiler are aware of, has to be defined in the corresponding
package. The lexer is divided into three parts: the lexer itself, the vocabulary
and a reader package to provide input/output operations. The lexer is used
by the compiler to read in primitive descriptions.

\subsection{src/compiler/parser}
The parser is used to parse and interpret the primitive description. Implementation
is comprehensive, but can be simplified by studying the \textit{BNF} syntax.
about supported statements. Naming of parser elements is aligned to the
BNF syntax. To keep source files small, many parts are outsourced by means
of \texttt{separate} (Ada keyword). The parser generates the following packages, if the input was \texttt{examples/threefish256.adb}. They are stored in \texttt{src/function}.\\ \\

\vfill

\begin{lstlisting}[caption=dxpl.ads]
with Interfaces;
package DXPL is 
  type Word is new Interfaces.Unsigned_64;
  subtype Rounds is Positive range 1 ..  9;
end DXPL; 
\end{lstlisting}

\clearpage

\begin{lstlisting}[caption=dxpl-support-state.ads]
package DXPL.Support.State is
  type Round_State is new Basic_State with
    record
      Message : Word_Array_4 := (others => 16#0#);
      key : word_array_4;
    end record;
    
  function Get_Name return String;
  function Get_Digest (State : in Round_State) return Word_Array;
  
  procedure Set_Digest (State : in out Round_State; Item : in Word_Array);
end DXPL.Support.State; 
\end{lstlisting}

\vfill

\begin{lstlisting}[caption=dxpl-support-state.adb]
package body DXPL.Support.State is
  procedure Set_Digest (State : in out Round_State; Item : in Word_Array) is
  begin
    State.Message := Item;
  end Set_Digest;
  
  function Get_Digest (State : in Round_State) return Word_Array is
  begin
    return State.Message;
  end Get_Digest;
  
  function Get_Name return String is
  begin
    return "Threefish 256";
  end Get_Name;
end DXPL.Support.State; 
\end{lstlisting}

\vfill

\begin{lstlisting}[caption=dxpl-darxplorer.adb]
with DXPL.Types;  use DXPL.Types;

procedure DXPL.DARXplorer is
  Threshold : Conditioned_Float := 2.0 ** (- (Integer (256)));
  Number_of_Modules : Integer := 17;
  Name : String := "Threefish 256";
  
  procedure Process is separate;

begin
  Process;
end DXPL.DARXplorer;
\end{lstlisting}

\clearpage

As you can see, listings 1 to 4 contain just a minimum of information
needed to compile the framework. The architecture of DARXplorer
is very generic. Up until now, we were missing the actual representation
of the cryptographic primitive. The package \texttt{dxpl-round\_function.adb}
has its implementation.

The procedure \texttt{DXPL\_Process} in the primitive description is
evaluated statement per statement. Each time, the parser encounters
a mathematical statement, it will build a corresponding syntax tree. The
tree is used to detect additions modulo $2^{32}$ or $2^{64}$. Each addition
has to be analyzed by one of the techniques we declared in advance
(see chapter \ref{chapter:techniques}). A problem occurred once, if
more than one addition is included in the mathematical statement. Here,
we have to examine each addition on its one. The syntax tree helps us
to create temporary statements of the initially expression. Each sole
statement is encapsulated in its own function.

Let's have a look at the main parts of the resulting \texttt{dxpl-round\_function.adb},
which are self-explanatory.

\begin{lstlisting}[label=lst:five,caption=dxpl-round\_function.adb]
package body DXPL.Round_Function is

  function Process_1(State : access Rnd_State) return access Word is
  begin
    Analyse (State.Message(0), State.Message(1), State.Message(0),
      State.Output_Candidates, State.Local_Probability);
    return State.Message(0)'Access;
  end;

  [..]

  function Process_17 (..) return access Word is
  begin
    [..]
  end;

  function Get_Message (Index : in Positive)
    return Word_Array is separate;
  function Get_Digest  (Index : in Positive)
    return Word_Array is separate;
  function Get_Key     (Index : in Positive)
    return Word_Array is separate;
  procedure Initialize (State : access Round_State) is separate;
  procedure Finalize (State : access Round_State) is separate;
  function  Number_of_Tests return Positive is separate;

begin
  Module_Sequence :=
      (1 => Process_1'Access, [..], 17=> Process_17'Access);
  Test_Vectors.Batch :=
      new Validation_Array'(1 => (Message => new Word_Array'([..]), 
                                  Digest => new Word_Array'([..])));
end;
\end{lstlisting}

\subsection{tests}
Test cases are produced in many domains according to the in EBNF
declared syntax presented in chapter \ref{chapter:syntax}. We provided
a \textit{Makefile} in each subdirectory. Tests are based on \textit{tg} \cite{Spiegel:tg},
a simple test driver generator for Ada. Unfortunately, \textit{tg} fails while processing, 
if more than one test raises an exception. Distinct tests with each example
were correctly executed.
To run a test, go into a subdirectory and type\\ \\
\noindent\hspace*{12mm}%
\texttt{make testgen}\\
\noindent\hspace*{12mm}%
\texttt{make compiler}\\
\noindent\hspace*{12mm}%
\texttt{make run}\\

\section{Supported syntax in DARXplorer}\label{chapter:syntax}
Cryptographic primitives are written in Ada, but we limit the powerfulness of
the programming language to simplify our parser. At the moment,
we support the following syntax (shown in EBNF). No guarantee of completeness.

\begin{lstlisting}[language=bash,caption=Supported syntax in DARXplorer]
main_program ::=
  { <with-declaration> [ <use-declaration> ] } <package-declaration>
  { <procedure-declaration> | <function_declaration>  | 
    <variables-declaration> } begin { <begin-declaration> } end ";"

with_declaration ::= 
  with <identifier> { "." <identifier> } ";"
 
numeric ::=
  { "0" .. "9" }
  
identifier ::= (simplified)
  "A" .. "Z" | "a" .. "z" | "_" | <numeric>

use-declaration ::= 
  use <identifier> { "." <identifier> } ";"
  
package_declaration ::=
  package body <identifier> is

procedure_declaration ::=
  procedure <identifier> is | procedure <identifier> 
  "(" <procedure_parameter_declaration> ")" is
  begin <statements> end [ <identifier> ] ";"

proc_parameter ::=
  <identifier> { "," <identifier } ":" [ in | out | in out ] 
  <identifier> [ := <expression> ]

procedure_parameter_declaration ::= 
  proc_parameter { ";" proc_parameter }

expr_element ::=
  <simple_expression> [ [ "=" | "/=" | "<" | "<=" | ">" | ">=" ] 
  <simple_expression> ] [ [ and | or | xor ] <simple_expression> ]

primary ::=
  "(" <expression> ")" | 
  <identifier> [ "(" <expression> { "," <expression> } ")" ]
  
simple_expression_support ::= 
  <primary> [ "**" <primary> ] [ "*" | mod | "/" ] [ "+" | "-" ]

simple_expression ::=
  [ "+" | "-" | "not" ] <simple_expression_support>
  { <simple_expression_support> }

expression ::=
  <expr_element> { <expr_element> }
  
statements ::=
  null ";" | return <expression> ";" | <for_loop_declaration> ";" | 
  <identifier> ":=" <expression> ";" | <declare_block_declaration>
   ";" | <identifier> "(" <expression> { "," <expression> ")" ";"

for_loop_declaration ::=
  for <identifier> <numeric> ".." <numeric> loop <statements> 
  end loop ";"

decl_item ::=
  <numeric> "=>" <numeric>
  
decl_element ::=
  <numeric> | <identifier> | "(" ( others "=>" ( <numeric> | 
  <identifier> )) | <decl_item> { "," <decl_item> } ")" 

variables_declaration ::=
  <identifier> { "," <identifier> } ":" [ constant ] <identifier>
  [ ":=" <decl_element> ] ";"

declare_block_declaration ::=
  declare <variables_declaration> begin <statements> end ";" 

function_declaration ::=
  function <identifier> is | function <identifier> 
  "(" <function_parameter_declaration> ")"
  return <identifier>  is begin <statements> end [ <identifier> ]";"

func_parameter ::=
  <identifier> { "," <identifier } ":" [ in ] <identifier>
  [ := <expression> ]

function_parameter_declaration ::= 
  <func_parameter> { ";" <func_parameter> }

var_item ::=
  <numeric> "=>" <numeric>

var_element ::=
  <numeric> | <identifier> | "(" ( others "=>" ( <numeric> | 
  <identifier> )) | <var_item> { "," <var_item> } ")" 

variables_declaration ::=
  <identifier> { "," <identifier> } ":" [ constant ] <identifier> 
  [ ":=" <var_element> ] ";"

begin_declaration ::=
  begin ( <configuration_declaration>, { test_vector_declaration } )
  end [ <identifier> ] ";"

configuration_declaration ::= 
  [ <identifier> "." ] <identifier> := "(" 
  [ DXPL_ALGORITHM => ] <identifier> "." <identifier> "," 
  [ DXPL_ROUNDS    => ] { <numeric> } ")" ","
  [ DXPL_TERMINATION => ] { <numeric> } ")" ";"

test_vector_declaration ::=
  [ <identifier> "." ] <identifier> := "("
  [ DXPL_Message =>  ] <array_declaration> ","
  [ [ DXPL_Key   =>  ] <array_declaration> "," ]
  [ DXPL_DIGEST  =>  ] <array_declaration> ")" ";"

array_element ::=
  [ <numeric>  => ] <numeric>

array_declaration ::= 
  "(" array_element { "," array_element } ")"
\end{lstlisting}

\section{Techniques for Analyzation}\label{chapter:techniques}

As it turned out, the Achilles heel of these primitives was some 
vulnerability to differential cryptanalysis, mostly based on XOR 
differences. When analysing a given primitive of this type, searching 
for good xor differences appears to be the way to go.

The toolbox will implement four different techniques to find "good" 
characteristics. Usually, the differential behaviour of a low 
Hamming-weight difference is better predictable (i.e., with a higher 
probability) than the behaviour of a "random" difference. We just start
with such a low Hamming-weight difference (1, 2, 3 or 4) and try all
possible input differences with the chosen Hamming-weight.

\subsection{Implemented techniques}

\begin{describe}{{\em Pedantic Petra\/}:}
\item[{\em Lazy Laura\/}:]
Laura just "guesses" that the the addition behaves like the xor, 
difference-wise, i.e., there are no carry bits generated. Starting with 
low Hamming-weight differences, it is very easy for her to compute the 
output difference (running the primitive in forward direction). The 
benefit of Laura’s approach is that it is very efficient and can be 
implemented for any number of rounds (for 80 rounds, or for 800 rounds).
  
\item[{\em Greedy Grete\/}:]
Laura just assumes no carry bits are generated. This is likely to 
exclude many "good" characteristics – a carry bit in one round leads 
to another difference in another round and may ultimately lead to a 
better (more probable) difference at the end. Grete optimises locally. 
Given some round’s input differences in some round, she enumerates all 
the most probable output differences. All of them are considered as 
the input difference for the next round. If there is more than one 
optimal difference, Grete’s run time will grow exponentially with the 
number of rounds.

\item[{\em Pedantic Petra\/}:]
Petra goes a step further than Grete. Given an input difference, she 
enumerates all output differences with a nonzero probability. Recall 
that this is feasible as long as the Hamming-weight of the difference 
is low. As the number of possible differences is growing fast, and 
the Hamming-weight also grows with the number of rounds, this can be 
done only for a few rounds. Petra is pedantic, but not stupid. A naive 
approach would make her deal with a huge number of extremely 
low-probable characteristics. But Petra sorts her intermediate results 
(characteristics with less than the desired number of rounds) by 
their probability. She always proceeds with the most probable 
intermediate result. Specifically, when she comes to the last round 
she just uses Grete’s approach.

\item[{\em Fuzzy Fiona\/}:]
Fiona's strategy is a new cryptoanalytical approach. Assuming that
the input for a S-Box, i.e. for a non-linear function, is \texttt{100101}.
The S-Box describes a mapping from 6 bits to 4 bits. However, 
the output can only be defined exactly for particular bits: \texttt{1??0},
whereby question marks represent bits, whose value cannot be
clearly fixed. Even so, one can specify the probability, that a bit
equals \texttt{1}. Now, rewriting the output with probabilities results
in \texttt{($1$, $\frac{1}{4}$, $\frac{2}{3}$, $0$)}. The third bit 
takes on value \texttt{1} with probability $\frac{2}{3}$, and the last bit
is definitly \texttt{0} and is never equals \texttt{1}.
Pursuing such probabilities while computing a hash digest, let one
employ predictions about the output, if the probability is distinct from $\frac{1}{2}$.
\end{describe}

\subsection{Adding new techniques}
Techniques are defined in packages named \texttt{dxpl-technique-*}.
Each of the them should implement a procedure with the following parameter list:

\begin{lstlisting}[language=Ada]
    X, Y : in Word;
    Z    : in out Word;
    Output_Candidates : in out Word_Vector.Vector;
    Probability       : in out Conditioned_Float;
\end{lstlisting}

to support the integration of a particular technique in the framework. 
The procedure serves as a parameter to the generic round function 
declared in \\ \texttt{dxpl-round\_function.adb}. See procedure
\texttt{Analyse} in listing \ref{lst:five}.

Listings \ref{lst:trudy} and \ref{lst:trudy2} show the default behavior, that should be implemented
when declaring a new procedure to overwrite \texttt{Analyse}. Here,
both terms of the sum were just added up.

\vfill

\begin{lstlisting}[label=lst:trudy,language=ada,caption=dxpl-technique-trudy.ads]
with DXPL.Round_Function;
with DXPL.Toolbox;
with DXPL.Toolbox.Trudy;
with DXPL.Types;  use DXPL.Types;

pragma Elaborate_All (DXPL.Round_Function);
pragma Elaborate_All (DXPL.Toolbox.Trudy);

generic
  Number_of_Modules : Integer;
package DXPL.Technique.Trudy is

  procedure Solve
    (X, Y				: in     Word;
     Z					: in out Word;
     Output_Candidates : in out Word_Vector.Vector;
     Probability       : in out Conditioned_Float);
  
  package Characteristic_Function is new DXPL.Round_Function
    (Analyse 			=> Solve,
     Number_of_Modules => Number_of_Modules);

  package DARX is new Toolbox
    (Round_Characteristic => Characteristic_Function,
     Toolbox_Name		  => "Trusty Trudy",
     Toolbox_Threshold    => Toolbox_Threshold,
     Logging_Formatter    => Default_Formatter);

  package DARXplorer is new DARX.Trudy;
  Toolbox : DARXplorer.Instance;
end DXPL.Technique.Trudy;
\end{lstlisting}

\vfill

\begin{lstlisting}[label=lst:trudy2,language=ada,caption=dxpl-technique-trudy.adb]
package body DXPL.Technique.Trudy is
  procedure Solve
    (X, Y				: in     Word;
     Z					: in out Word;
     Output_Candidates : in out Word_Vector.Vector;
     Probability       : in out Conditioned_Float) is
    pragma Warnings (Off, Output_Candidates);
    pragma Warnings (Off, Probability);
  begin
    Z := X + Y;
  end Solve;
end DXPL.Technique.Trudy;
\end{lstlisting}

\clearpage

It is just that simple. If you had to implement the behavior of
\textit{Lazy Laura}, you need to represent the addition by
xor and compute the probability, see listing \ref{lst:laura} for
a solution.

\begin{lstlisting}[label=lst:laura,language=ada,caption=dxpl-technique-laura.adb]
with DXPL.Support.Differential_Addition;
use  DXPL.Support.Differential_Addition;

package body DXPL.Technique.Laura is

  function "+" (L, R : Word) return Word is
  begin
    return L xor R;
  end "+";

  procedure Compute_Optimal_Output
    (X, Y				: in     Word;
     Z					: in out Word;
     Output_Candidates : in out Word_Vector.Vector;
     Probability       : in out Conditioned_Float) is
    pragma Warnings (Off, Output_Candidates);
  begin
    Z := X + Y;
    Differential_Probability (X, Y, Z, Probability);
  end Compute_Optimal_Output;
end DXPL.Technique.Laura;
\end{lstlisting}

\section{Provide a user-defined format for log files}\label{chapter:log}

Analyzation produces suitable log files. Formatting is user-defined
and can be changed by subclassing \texttt{dxpl.io.formatter.ads}.
Per default, a CSV format is chosen to log each stage and round, i.e.:

\begin{tabbing}
;Round; Stage; Global Probability; Local Probability; Hamming Weight; Differentials;\\
\\
; 1; 1;     ;           5.0000E-0  [2\^\,(-1)]; 1; \\
;;;;;;0000\_0000\_0000\_0008; \\
;;;;;;0000\_0000\_0000\_0008; \\
;;;;;;0000\_0000\_0000\_0000; \\
;;;;;;0000\_0000\_0000\_0000; \\
\end{tabbing}

Probabilities are presented in decimal with four decimal places and
rounded to the power of two, which increases readability. Words are
represented with leading zeros and underscores every fourth digit.

\section{Challenges and future work}\label{chapter:future}

DARXplorer made a significant leap compared to its predecessor. The framework
can handle Ada-based inputs to allow the declaration of arbitrary cryptographic
primitives based on a mixture of addition, rotation and xor. We showed, that several
primitives can be declared to be compatible with DARXplorer. Nevertheless,
the syntax DARXplorer currently understands is very limited. 

Producing examples for DARXplorer showed, that converting cryptographic
primitives to Ada is a error prone task. Most primitives are declared in \textit{C}. It 
might be helpful to replace the parser for Ada with a parser, 
that converts C code in Ada. In this way, primitives could be applied to DARXplorer
by means of simple copy and paste.

Then again, declaration of primitives in Ada allows designers to fall back upon
the powerfulness of Ada in scopes, where DARXplorer does not need to analyze the 
source code. These sections are variable declarations, additional functions and
procedures outside from \texttt{DXPL\_Initialize}, \texttt{DXPL\_Finalize} and
\texttt{DXPL\_Process}.

A major drawback of DARXplorer 2.0 as opposed to its predecessor is, that the framework
no longer supports backward analyzing. First tests showed that maximum
rounds, that could be reached with the former DARXplorer by starting in the middle 
and perform both forward and backward analyze, has been bisected. To support 
backward analyzation, one can either declare a backward round of the primitive
by its own or, much better, find a way to automatically reverse the forward 
declaration within DARXplorer itself.

Further on, it would help matters, if the framework runs distributed on several cores.
At the moment, only one core is occupied with analyzation. Each analyzation involves
pursue an input difference through as much rounds as possible. For each input
difference, an extra thread could be created.

\bibliographystyle{acmtrans} % natdin
\bibliography{acmtrans}

\end{document}


